%%%& -job-name=CV_Andres_Navarrete
\documentclass[11pt,letterpaper,sans,english]{moderncv}        % possible options include font size ('10pt', '11pt' and '12pt'), paper size ('a4paper', 'letterpaper', 'a5paper', 'legalpaper', 'executivepaper' and 'landscape') and font family ('sans' and 'roman')
\moderncvstyle{classic}                             % style options are 'casual' (default), 'classic', 'oldstyle' and 'banking' and 'fancy'
\moderncvcolor{purple}                               % color options 'blue' (default), 'orange', 'green', 'red', 'purple', 'grey' and 'black'
\nopagenumbers{}                                  % uncomment to suppress automatic page numbering for CVs longer than one page
\usepackage[utf8]{inputenc}                       % if you are not using xelatex ou lualatex, replace by the encoding you are using
\usepackage[scale=0.85,letterpaper]{geometry}
\usepackage{babel}
\usepackage{fontawesome}

%----------------------------------------------------------------------------------
%            personal data
%----------------------------------------------------------------------------------
\firstname{Andrés}
\familyname{Navarrete Contardo}
%\title{Currículum Vitae}                               
\address{Santiago, Chile}  %\address{street and number}{postcode city}{country}      
\phone[mobile]{+56 9 85897103}                          
                  
%\homepage{www.johndoe.com}
\social[linkedin]{andresnavarretecontardo}
\social[github]{AndresNavarrete}
\email{ainavarrete@uc.cl}
%\email{a.navarrete.contardo@gmail.com}

% \photo[64pt][0.4pt]{picture}                

% \quote{
% Ingeniero Civil Industrial Transporte // Investigación en logística e investigación de operaciones | Experiencia profesional en logística y consultoría operacional y estratégica.
% }

\begin{document}
\makecvtitle
\section{Experiencia profesional}

\cventry{2023 - Actual}{\href{https://www.linkedin.com/company/metro-de-santiago-s-a-/}{Metro de Santiago}}{}{}{}{
\begin{description}
\item [Data Scientist] Responsable de analizar e interpretar grandes volúmenes de datos para respaldar decisiones estratégicas en la organización. Desarrollo modelos analíticos y predictivos utilizando técnicas estadísticas y de Machine Learning. Gestiono bases de datos, automatizo reportes y optimizo procesos mediante la integración de datos provenientes de múltiples sistemas. Además, mi labor incluye la creación de visualizaciones y la generación de \textit{insights} accionables que contribuyen a mejorar la eficiencia operativa y la experiencia de los usuarios.
\end{description}
}


\cventry{2021 - 2023}{\href{https://www.linkedin.com/company/dispatchtrack-es/}{Beetrack - DispatchTrack}}{}{}{}{
\begin{description}
\item [Data Engineer (2022 - 2023)] Ejecuté proyectos de ingeniería y análisis de datos, diseñando soluciones para la automatización de procesos y la extracción de valor de grandes volúmenes de información. Principales logros:
\begin{itemize}
\item Diseñé e implementé integraciones entre sistemas comerciales, financieros y operativos, administrando el Datawarehouse corporativo y gestionando la infraestructura de transferencia de datos.
\item Generé visibilidad de métricas clave mediante reportes automatizados en Tableau, integrados con herramientas como Hubspot e Intercom, para prevenir la fuga de clientes.
\item Implementé modelos predictivos para segmentación de clientes y análisis de comportamiento, asegurando su funcionamiento autónomo en producción con prácticas de MLOps.
\end{itemize}
\item [Ingeniero de Proyectos (2021)] Desarrollador en el equipo de producto, enfocado en crear soluciones tecnológicas para optimizar operaciones logísticas. Principales proyectos:
\begin{itemize}
\item Diseñé y programé algoritmos para ruteo de flotas, logrando reducir costos y aumentar la eficiencia operativa, además de un modelo de gestión de inventarios que optimizó costos en industrias de alta rotación.
\item Construí un servicio predictivo basado en inteligencia artificial para estimar tiempos de viaje considerando condiciones de tráfico, mejorando la planificación y la toma de decisiones en operaciones logísticas.
\end{itemize}
\end{description}
}


\cventry{2020 - 2021}{\href{https://www.dictuc.cl/}{Dictuc}}{}{}{}{
\begin{description}
\item [Ingeniero de Proyectos] Como consultor en estudios de demanda de transporte para el Banco Interamericano de Desarrollo, fui responsable del diseño e implementación de encuestas de movilidad en Santiago y Bogotá, así como del análisis de datos masivos y la calibración de modelos estadísticos para estimar la demanda de transporte. También desarrollé estudios que contribuyeron a una mejor comprensión de los patrones de movilidad en América Latina y el Caribe. \href{https://blogs.iadb.org/transporte/es/por-que-en-america-latina-y-el-caribe-usamos-tanto-el-automovil/}{Ver estudio aquí}.
\end{description}
}




\section{Educación}
\cventry{2018 - 2020}{Magíster en Ciencias de la Ingeniería}{Pontificia Universidad Católica de Chile}{}{}{Tesis en Departamento de Ingeniería en Transporte y Logística: \textit{Planificación eficiente de una operación de despacho a domicilio integrada a servicios de transporte público}.  \href{https://repositorio.uc.cl/xmlui/handle/11534/65009}{\textbf{Ver tesis aquí.}}}{} 

\cventry{2013 - 2019}{Ingeniería Civil Industrial}{PUC}{}{}{Diploma en Ingeniería en Transporte.}

\cventry{2013 - 2016}{Licenciatura en Ciencias Naturales y Matemáticas}{PUC}{College UC}{}{Major: Investigación Operativa. Minor: Recursos Humanos \& Sistemas de Transporte.}

\newpage


\subsection{Formación profesional}
\cventry{}{Certificaciones}{}{}{}{
    \begin{itemize}
        \item Tableau avanzado para Data Science. \href{https://www.udemy.com/certificate/UC-1467d971-22b4-47fc-9e4e-f8ee401bbb0d/}{\textbf{Ver certificado en Udemy}}
		\item Algoritmos de machine learning aplicado en Python y R. \href{https://www.udemy.com/certificate/UC-33369609-4adb-45a7-b550-d2e5d252a04c/}{\textbf{Ver certificado en Udemy}}
		\item Curso aplicado de Apache Airflow. \href{https://www.udemy.com/certificate/UC-af5ee666-1176-4f27-a834-7b7cf699eee8/}{\textbf{Ver certificado en Udemy}}
		\item Machine learning aplicado a plataformas y servicios de AWS. \href{http://www.coursera.org/verify/VHZPPXDH7N3N}{\textbf{Ver certificado en Coursera}}
		\item Certificación Google Analytics 4. \href{https://skillshop.exceedlms.com/student/award/MhNWrQWjRLbmND5DzQhp7X4p}{\textbf{Ver certificado en Google}}
	\end{itemize}
	}


\section{Experiencia académica}
\cventry{2020}{Expositor en congresos}{}{}{}{Expositor en IV Congreso de Estudiantes de Ingeniería USM-UC, realizado del 27 al 28 de Agosto del 2020.}

\cventry{2015-2019}{Ayudante Docente}{Facultad de Ingeniería}{PUC}{}{\begin{itemize}
		\item Magíster de Ingeniería Industrial: Modelos de Simulación 
		\item Ingeniería Industrial y de Sistemas: Optimización, Modelos Estocásticos, Simulación, Capstone de Investigación Operativa (4 veces) y Gestión de Operaciones.
		\item Ingeniería de Transporte y Logística: Flujo en redes
		\item Cursos en inglés: Research, Innovation and Entrepreneurship.
	\end{itemize}}
\cventry{2015-2017}{Ayudante Docente}{Facultad de Matemáticas}{Pontificia Universidad Católica de Chile}{}{Álgebra, Cálculo I, II, III y Ecuaciones Diferenciales}


\section{Voluntariados}
\cventry{2021-2023}{Tutor académico}{Formando Chile}{}{}{Tutor académico de estudiantes de primeros años de educación superior. Imparto clases y brindo apoyo en cursos de matemáticas y programación.}

\cventry{2014-2018}{Profesor y Coordinador}{PrePreu}{}{}{Coordinador General de proyecto de educación social con más de 170 estudiantes de colegios vulnerables de la Región Metropolitana. Fui profesor por cinco años y coordinador general durante dos años. Gestioné un equipo de más de 50 profesores voluntarios.}

\cventry{2016-2019}{Tutor Académico}{College UC}{}{}{Apoyo académico y seguimiento de estudiantes de vías de admisión especial.}


\section{Reconocimientos}
\cventry{2017}{100 Jóvenes Líderes}{El Mercurio: Revista "El Sábado", Universidad Adolfo Ibáñez}{}{}{Reconocimiento por proyecto de innovación en seguridad de cuerpo de bomberos y brigadistas en Chile. Este premio se otorga anualmente a 100 chilenos menores de 35 años que demuestran ser potenciales líderes para el pais.}{}{}
\cventry{2017}{Finalista Global Student Entrepreneurship Challenge}{Virgnia Tech}{Virginia, USA}{}{Finalista de concurso internacional de emprendimiento universitario.}
\cventry{2017}{Mejor alumno promoción 2017}{College UC}{Licenciatura en Ciencias Naturales y Matemáticas}{}{Estudiante con la calificación más alta.}

\section{Competencias técnicas}
\cvitemwithcomment{Inglés}{Alto. 99 en TOEFL IBT, Octubre 2019}{}
\cvitemwithcomment{Tecnologías}{Python, Linux, AWS, R, Kotlin, Gurobi, Latex}{ \href{https://github.com/AndresNavarrete}{\textbf{Ver portafolio en GitHub}}  }
%\section{Referencias}
  %\cventry{}{José Francisco Caiceo}{jfcaiceo@beetrack.com}{CTO de Beetrack}{}{}
 % \cventry{}{Paula Iglesias}{paiglesi@uc.cl}{Jefa de proyectos en DICTUC}{}{}
  %\cventry{}{Mathias Klapp}{maklapp@ing.puc.cl}{ Profesor Asistente PUC}{Profesor guía en Tesis de magister}{}
%  \cventry{}{Felipe Sandoval}{Felipe.Sandoval@Walmart.com}{ Gerente de Logística E-Commerce Walmart Chile}{Jefe drecto en práctica profesional}{}

%-----       letter       ---------------------------------------------------------

\end{document}
