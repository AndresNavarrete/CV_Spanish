%%%& -job-name=CV_Andres_Navarrete
\documentclass[11pt,letterpaper,sans,english]{moderncv}        % possible options include font size ('10pt', '11pt' and '12pt'), paper size ('a4paper', 'letterpaper', 'a5paper', 'legalpaper', 'executivepaper' and 'landscape') and font family ('sans' and 'roman')
\moderncvstyle{classic}                             % style options are 'casual' (default), 'classic', 'oldstyle' and 'banking' and 'fancy'
\moderncvcolor{purple}                               % color options 'blue' (default), 'orange', 'green', 'red', 'purple', 'grey' and 'black'
\nopagenumbers{}                                  % uncomment to suppress automatic page numbering for CVs longer than one page
\usepackage[utf8]{inputenc}                       % if you are not using xelatex ou lualatex, replace by the encoding you are using
\usepackage[scale=0.85,letterpaper]{geometry}
\usepackage{babel}
\usepackage{fontawesome}

%----------------------------------------------------------------------------------
%            personal data
%----------------------------------------------------------------------------------
\firstname{Andrés}
\familyname{Navarrete Contardo}
%\title{Currículum Vitae}                               
\address{Santiago, Chile}  %\address{street and number}{postcode city}{country}      
\phone[mobile]{+56 9 85897103}                          
                  
%\homepage{www.johndoe.com}
\social[linkedin]{andresnavarretecontardo}
\social[github]{AndresNavarrete}
%\email{ainavarrete@uc.cl}
\email{a.navarrete.contardo@gmail.com}

% \photo[64pt][0.4pt]{picture}                

% \quote{
% Ingeniero Civil Industrial Transporte // Investigación en logística e investigación de operaciones | Experiencia profesional en logística y consultoría operacional y estratégica.
% }

\begin{document}
%-----       resume       ---------------------------------------------------------
\makecvtitle

\section{Experiencia profesional}
\cventry{2022 - Actual}{Data Scientist}{Beetrack}{}{}{Desarrollo soluciones tecnológicas basadas en grande volúmenes de información. Trabajo en dos tipos de proyectos. (1) Machine Learning aplicado a productos de la empresa. Segmentación de clientes, automatización de acciones de ventas y recomendación de productos. Todos estos proyectos requieren desarrollar algoritmos de inteligencia artificial e implementarlos en servicios \textit{cloud}. (2) Análisis de datos e inteligencia de negocios basado en grandes volúmenes de información. Las principales tecnologías aplicadas son Airflow, Python, Tableau y AWS EC2.}


\cventry{2021 - 2022}{Ingeniero de proyectos}{Beetrack}{}{}{Desarrollador en el equipo de Labs. Diseñamos e implementamos soluciones tecnológicas para los diversos productos de Beetrack.  Trabajamos con dos tipos de proyectos. (1) Desarrollo de algoritmos, donde destacan los proyectos de diseño y desarrollo de un servicio para la administración de inventarios y para el ruteo de flotas de vehículos. (2) Soluciones de Data Science, donde se destaca el desarrollo de un servicio predicativo de tiempos de viaje considerando tráfico usando Machine Learning. Las principales tecnologías aplicadas son Kotlin, Python y AWS EC2.}

\cventry{2020}{Ingeniero de proyectos}{DICTUC}{}{}{Consultor de estudios de demanda de transporte. Junto al equipo de proyectos trabajé en el desarrollo, ejecución y entrega de dos proyectos para la estimación de demanda por transporte. (1) Hicimos un estudio demanda por uso de automóviles en ciudades de América Latina y el Caribe para el Banco Interamericano de Desarrollo. Mis responsabilidades fueron diseñar e implementar encuestas de movilidad en Santiago y Bogotá, analizar bases de datos, calibrar modelos estadísticos y generar reportes. (2) Desarrollamos un estudio para caracterizar la demanda por transporte público de personas con discapacidad física o mental en Chile. Mis responsabilidades fueron implementar y gestionar la aplicación de encuestas en seis capitales regionales de Chile. Las principales tecnologías aplicadas fueron Python, Qualtrics y R.}
\cventry{2019}{Asesoría}{Pyme UC}{}{}{Ayudante en asesoría de gestión de negocios y administración a empresa del rubro de la construcción. Desarrollamos un nuevo modelo de negocios para profesionalizar la empresa. Definimos la visión y misión de la empresa y ordenamos los estados de resultados.}

\cventry{2018}{Práctica profesional}{Walmart Chile}{Logística y Supply Chain}{}{Miembro de proyecto de mejoramiento y modernización en logística de despacho a domicilio. Trabajé junto a diversas consultoras operacionales para seleccionar a un nuevo proveedor de servicios.}

\section{Educación}
\cventry{2018-2020}{Magíster en Ciencias de la Ingeniería}{Pontificia Universidad Católica de Chile}{}{}{Tesis en Departamento de Ingeniería en Transporte y Logística: "Planificación eficiente de una operación de despacho a domicilio integrada a servicios de transporte público"}{ }

\cventry{2013-2019}{Ingeniería Civil Industrial}{PUC}{}{}{Diploma en Ingeniería en Transporte.}

\cventry{2013-2016}{Licenciatura en Ciencias Naturales y Matemáticas}{PUC}{College UC}{}{Major: Investigación Operativa. Minor: Recursos Humanos \& Sistemas de Transporte.}

\newpage


\subsection{Formación profesional}
\cventry{}{Cursos online}{}{}{}{\begin{itemize}
		\item Curso  de algoritmos de machine learning aplicado en Python y R. \href{https://www.udemy.com/certificate/UC-33369609-4adb-45a7-b550-d2e5d252a04c/}{\textbf{Ver certificado en Udemy}}
		\item Curso SENCE online metodologías ágiles de trabajo para el desarrollo de productos en BCN School
		\item Curso online sobre machine learning aplicado a plataformas y servicios de AWS. \href{http://www.coursera.org/verify/VHZPPXDH7N3N}{\textbf{Ver certificado en Coursera}}
	\end{itemize}}


\section{Experiencia académica}
\cventry{2020}{Expositor en congresos}{}{}{}{Expositor en IV Congreso de Estudiantes de Ingeniería USM-UC, realizado del 27 al 28 de Agosto del 2020.}

\cventry{2015-2019}{Ayudante Docente}{Facultad de Ingeniería}{PUC}{}{\begin{itemize}
		\item Magíster de Ingeniería Industrial: Modelos de Simulación 
		\item Ingeniería Industrial y de Sistemas: Optimización, Modelos Estocásticos, Simulación, Capstone de Investigación Operativa (4 veces) y Gestión de Operaciones.
		\item Ingeniería de Transporte y Logística: Flujo en redes
		\item Cursos en inglés: Research, Innovation and Entrepreneurship.
	\end{itemize}}
%\cventry{2015-2017}{Ayudante Docente}{Facultad de Matemáticas}{Pontificia Universidad Católica de Chile}{}{Álgebra, Cálculo I, II, III y Ecuaciones Diferenciales}


\section{Voluntariados}
\cventry{2021 - Actual}{Tutor académico}{Formando Chile}{}{}{Tutor académico de estudiantes de primeros años de educación superior. Imparto clases y brindo apoyo en cursos de matemáticas y programación.}

\cventry{2014-2018}{Profesor y Coordinador}{PrePreu}{}{}{Coordinador General de proyecto de educación social con más de 170 estudiantes de colegios vulnerables de la Región Metropolitana. Fui profesor por cinco años y coordinador general durante dos años. Gestioné un equipo de más de 50 profesores voluntarios.}

\cventry{2016-2019}{Tutor Académico}{College UC}{}{}{Apoyo académico y seguimiento de estudiantes de vías de admisión especial.}


\section{Reconocimientos}
\cventry{2017}{100 Jóvenes Líderes}{El Mercurio: Revista "El Sábado", Universidad Adolfo Ibáñez}{}{}{Reconocimiento por proyecto de innovación en seguridad de cuerpo de bomberos y brigadistas en Chile. Este premio se otorga anualmente a 100 chilenos menores de 35 años que demuestran ser potenciales líderes para el pais.}{}{}
\cventry{2017}{Finalista Global Student Entrepreneurship Challenge}{Virgnia Tech}{Virginia, USA}{}{Finalista de concurso internacional de emprendimiento universitario.}
\cventry{2017}{Mejor alumno promoción 2017}{College UC}{Licenciatura en Ciencias Naturales y Matemáticas}{}{Estudiante con la calificación más alta.}

\section{Competencias técnicas}
\cvitemwithcomment{Inglés}{Alto. 99 en TOEFL IBT, Octubre 2019}{}
\cvitemwithcomment{Tecnologías}{Python, R, Kotlin, Gurobi, Latex, SIMIO}{}
\section{Referencias}
  \cventry{}{José Francisco Caiceo}{jfcaiceo@beetrack.com}{CTO de Beetrack}{}{}
  \cventry{}{Paula Iglesias}{paiglesi@uc.cl}{Jefa directa en DICTUC}{}{}
  \cventry{}{Mathias Klapp}{maklapp@ing.puc.cl}{ Profesor Asistente PUC}{Profesor guía en Tesis de magister}{}
%  \cventry{}{Felipe Sandoval}{Felipe.Sandoval@Walmart.com}{ Gerente de Logística E-Commerce Walmart Chile}{Jefe drecto en práctica profesional}{}

%-----       letter       ---------------------------------------------------------

\end{document}
